\section{Ponte H}

\subsection{Sobre a ponte H}

Os motores de corrente contínua trabalham nos dois sentidos de rotação quando invertidas suas polaridades. A ponte H é um circuito que tem a função de controlar esse sentido de rotação do motor a partir da inversão de sua polaridade. O exemplo a seguir, ilustra uma ponte H composta de quatro transistores que trabalham em pares nas diagonais. Basicamente, quando se aciona uma chave tem-se 12V e a outra chave-par leva o terra (0V) para o motor. A utilização de  NPN e PNP é aconselhável para evitar uma perda de tensão maior entre eles, dessa forma, a carga (motor) fica sempre ligado aos coletores dos transistores.