\section{Requisitos de Software}

\subsection{Web e Mobile}

Para os requisitos a seguir, o termo "aplicação" é usado para designar tanto aplicativo móvel quanto aplicação web.

\textbf{1. A aplicação deve exibir os dados capturados pelos sensores}

Os dados coletados pelos sensores instalados na estufa devem ser transmitidos às aplicações web e mobile, de forma que possam ser apresentados ao proprietário da estufa em tempo real.


\textbf{2. A aplicação deve permitir a troca de água do sistema.}

Embora a troca de água seja acionada automaticamente quando o nível de pH esteja fora da faixa aceitável, o usuário deve ser capaz de acionar esta funcionalidade manualmente.

\textbf{3. A aplicação deve exibir notificações quando o nível de pH da água estiver abaixo ou acima do nível ideal.}

Como um nível impróprio de pH da água é prejudicial para hortaliças, a aplicação deve exibir uma notificação sempre que, por algum motivo, o nível de pH fique acima ou abaixo do especificado como ideal. 

\textbf{4. A aplicação deve exibir notificações quando a temperatura estiver abaixo ou acima do nível ideal.}

A estufa deve ser mantida em uma faixa de temperatura especificada como ideal para as hortaliças. Sempre que, por algum motivo, a temperatura interna da estufa fique acima ou abaixo desta faixa, uma notificação deve ser exibida pela aplicação.

\textbf{5. A aplicação deve realizar a abertura/fechamento da gaveta por um comando.}

Em qualquer momento, o usuário poderá acionar a abertura/fechamento da gaveta pelo aplicativo ou pelo sistema web.


\section{Web Service}

A solução exige componentes capaz de manter comunicação unificada entre a estufa e as interfaces web e mobile. A necessidade de se atualizar parâmetros em tempo real orienta ao uso de uma solução com rabbitmq juntamente com o protocolo de comunicação AMQP. 

A API que será desenvolvida terá auxílio do microframework Flask, tendo este a responsabilidade pela integração entre os componentes de software. A solução da API seguirá o padrão arquitetural de microsserviços. 

\section{Linguagens e frameworks}

A construção do backend será desenvolvida em Python com o auxílio do framework Nameko, sendo este desenvolvido especialmente para aplicações de microsserviços. Além disso, será utilizado o microframework Flask para a implantação da API e a geração de rotas para requisição de páginas via AMQP, para que seja possível a conversão dos dados coletados em formatos em JSON, sendo este subsídio de entrada o frontend. \cite{rocha}

Para o frontend, serão utilizados o ReactJS para a aplicação web e o React Native na aplicação mobile. 

\section{Sistema mobile de controle}

A estufa poderá ser controlada via um aplicativo instalado no smartphone do proprietário da mesma. Isto será alcançado por meio de uma comunicação com o WebService citado acima.

Por meio do aplicativo, o proprietário poderá monitorar e alterar as condições internas da estufa de forma remota. Os dados coletados por meio dos sensores instalados no interior da estufa serão transmitidos ao smartphone do proprietário em tempo real. Quaisquer alterações definidas pelo aplicativo irão acionar os componentes relevantes para efetivar as mesmas.

O aplicativo mobile será desenvolvido utilizando o framework React Native. Esta tecnologia foi selecionada pois ela permite uma comunicação facilitada com o WebService definido acima.

\section{Sistema web de controle}

O sistema web oferece uma alternativa para o controle de estufa. Será desenvolvida em React JS, também se conectará ao WebService e oferecerá os mesmos mecanismos de controle disponíveis no Sistema Mobile de Controle.

\section{Histórias de Usuário}

Seguindo as recomendações do SCRUM, uma metodologia ágil de desenvolvimento de software, foram definidas histórias de usuário que representam funcionalidades planejadas do sistema. Cada história foi analisada pela equipe de desenvolvimento e pontuada segundo seu nível de complexidade. A seguir encontra-se a listagem delas. 

\textbf{US01 - Visualizar medições}

\textbf{Pontuação:} 3

\textbf{Eu como:} Usuário

\textbf{Desejo:} visualizar as medições feitas pelos sensores

\textbf{Para que:} possa ter um monitoramento sobre a hortaliça plantada

\begin{itemize}
	\item Exibir valores atuais de temperatura, potencial hidrogeniônico, nível de água, consumo de água, luminosidade.
	\item Exibir valores históricos
		\subitem Gráficos de linha
		\subitem Exibir dados referentes ao dia, semana, mês, ano
		\subitem Utilizar cores diferentes para cada grandeza
\end{itemize}

\textbf{US02 - Login de usuário}

\textbf{Pontuação:} 5

\textbf{Eu como:} Usuário

\textbf{Desejo:} logar no sistema

\textbf{Para que:} tenha acesso a modificações no ambiente interno da estufa.

\begin{itemize}
	\item Exibir tela de login.
	\item Utilizar endereço de email para login.
	\item Utilizar PIN para autenticação
		\subitem PIN de 4 dígitos
\end{itemize}

\textbf{US03 - Alertar usuário}

\textbf{Pontuação:} 2

\textbf{Eu como:} Sistema

\textbf{Desejo:} alertar o usuário sobre mudanças no ambiente interno da estufa

\textbf{Para que:} ele possa fazer as alterações necessárias

\begin{itemize}
	\item Exibir um alerta quando a temperatura estiver elevada ou abaixo do necessário
\end{itemize}

\textbf{US04 - Visualizar ambiente interno}

\textbf{Pontuação:} 13

\textbf{Eu como:} Usuário

\textbf{Desejo:} visualizar o ambiente interno da estufa

\textbf{Para que:} possa ter um acompanhamento sobre a hortaliça plantada

\begin{itemize}
	\item Mostrar a estufa em tempo real
\end{itemize}

\textbf{US05 - Troca de água}

\textbf{Pontuação:} 5

\textbf{Eu como:} Usuário

\textbf{Desejo:} acionar a troca da água do sistemaa

\textbf{Para que:} os nutrientes sejam renovados e a água com pH alterado seja descartada

\begin{itemize}
	\item Exibir um painel de controle para acionar a troca da água do sistema

\end{itemize}

\textbf{US06 - Visualizar relatório de temperatura d’água}

\textbf{Pontuação:} 3

\textbf{Eu como:} Usuário

\textbf{Desejo:} visualizar um relatório com a temperatura de água ao longo de um período específico

\textbf{Para que:} tenha controle sobre a temperatura de água em um período selecionado


\textbf{US07 - Visualizar relatório de temperatura ambiente interna da estufa}

\textbf{Pontuação:} 3

\textbf{Eu como:} Usuário

\textbf{Desejo:} visualizar um relatório com a temperatura ambiente interna da estufa ao longo de um período específico

\textbf{Para que:} tenha controle sobre a temperatura ambiente interna da estufa em um período selecionado


\textbf{US08 - Visualizar relatório do nível de pH da estufa}

\textbf{Pontuação:} 3

\textbf{Eu como:} Usuário

\textbf{Desejo:} visualizar um relatório com o nível de ph da água da estufa ao longo de um período específico

\textbf{Para que:} tenha controle sobre o nível de ph da água da estufa em um período selecionado


\textbf{US09 - Visualizar relatório da umidade interna da estufa}

\textbf{Pontuação:} 3

\textbf{Eu como:} Usuário

\textbf{Desejo:} visualizar um relatório com a umidade interna da estufa ao longo de um período específico

\textbf{Para que:} tenha controle sobre o nível da umidade interna da estufa em um período selecionado


\textbf{US10 - Alerta sobre falhas de comunicação}

\textbf{Pontuação:} 5

\textbf{Eu como:} Sistema

\textbf{Desejo:} alertar o usuário sobre falhas de comunicação com a estufa

\textbf{Para que:} o usuário fique ciente e possa tomar providências

\begin{itemize}
	\item Identificar origem da falha
	
\end{itemize}

\textbf{US11 - Abrir a gaveta da estufa}

\textbf{Eu como:} Usuário

\textbf{Desejo:} abrir a gaveta da estufa

\textbf{Para que:} realizar a coleta/inserção da planta ou embrião da planta.

\textbf{US12 - Fechar a gaveta da estufa}
	
	\textbf{Eu como:} Usuário
	
	\textbf{Desejo:} fechar a gaveta da estufa
	
	\textbf{Para que:} realizar a coleta/inserção da planta ou embrião da planta.


\section{Microsserviços utilizados no projeto}

Para o escopo da integração e das funcionalidades pertinentes ao sistema como um todo, foram definidos os seguintes microsserviços a serem desenvolvidos pela equipe:

\begin{itemize}
	\item Monitoramento de temperatura, e do nível hidrogeniônico da água.
	\item Monitoramento de temperatura ambiente e umidade relativa do ar, internos à água.
	\item Login do usuário.
	\item Gerenciamento de usuário.
\end{itemize}

\section{Diagramas}

\subsection{Diagramas de classe}

Os diagramas de classe representa a estrutura do sistema, em suma suas classes, atributos, métodos e relacionamentos. \cite{ibm}

Como a arquitetura de microsserviços prega o desacoplamento dos microsserviços, portanto, para cada microsserviço do projeto existe um diagrama pertinente àquele microsserviço em questão.

\subsubsection{Microsserviço de monitoramento de temperatura ambiente}

Na imagem abaixo, é possível observar como é a estrutura padrão para os microserviços que contam com a integração dos softwares (embarcados, webservice, cliente) como: monitoramento de temperatura, e do nível hidrogeniônico da água; e, monitoramento de temperatura ambiente e umidade relativa do ar;

\begin{figure}[H]
	\centering
	\includegraphics[width=17cm]{figuras/microservico_temperatura.png}
	\caption{Diagrama de classe do microsserviço de monitoramento de temperatura ambiente} \label{microservico_temperatura}
\end{figure}

\subsubsection{Microsserviço de gerenciamento do usuário}

Logo abaixo, há uma ilustração da estrutura das classes que contam com o microsserviço de gerenciamento do usuário:

\begin{figure}[H]
	\centering
	\includegraphics[width=17cm]{figuras/microservico_usuario.png}
	\caption{Diagrama de classe do microsserviço de gerenciamento do usuário} \label{microservico_usuario}
\end{figure}

Para o microsserviço de Login de usuário, o diagrama seguirá a mesma estrutura do diagrama acima.

\subsection{Diagrama de Sequência}

Diagramas de sequência podem ser utilizados para visualizar e avaliar o fluxo lógico de um sistema. Eles exibem a forma pela qual informações, ações e eventos são transmitidos entre todas as entidades relacionadas a uma funcionalidade específica.

A seguir, são apresentados os diagramas de sequência de algumas das funcionalidades chave do projeto Greenhouse.


\subsubsection{Enviar Alerta}

O fluxo do diagrama de sequência abaixo mostra as ações efetuadas pelos sistemas para realizar o envio de alertas ao usuário.

\begin{figure}[H]
	\centering
	\includegraphics[width=17cm]{figuras/enviar_alerta.png}
	\caption{Diagrama de sequência de enviar alerta} \label{enviar_alerta}
\end{figure}

\subsubsection{Exibir Relatório}

O fluxo do diagrama de sequência abaixo mostra as ações efetuadas tanto pelos sistemas para exibir o relatório das informações do ambiente da estufa (i.e. média de temperatura, umidade, iluminação, e volume de água utilizado) ao usuário.

\begin{figure}[H]
	\centering
	\includegraphics[width=18cm]{figuras/exibir_relatorio.png}
	\caption{Diagrama de sequência de exibir relatório} \label{exibir_relatorio}
\end{figure}

\subsubsection{Troca de Água Automática}

O fluxo do diagrama de sequência abaixo mostra as ações efetuadas pelos sistemas para realizar a troca automática de água.

\begin{figure}[H]
	\centering
	\includegraphics[width=17cm]{figuras/troca_agua.png}
	\caption{Diagrama de sequência de troca de água automática} \label{troca_agua}
\end{figure}

\subsubsection{Troca de Água Manual}

O fluxo do diagrama de sequência abaixo mostra as ações efetuadas pelos sistemas para realizar a troca manual de água.

\begin{figure}[H]
	\centering
	\includegraphics[width=18cm]{figuras/troca_agua_manual.png}
	\caption{Diagrama de sequência de troca de água manual} \label{troca_agua_manual}
\end{figure}

\subsection{Diagrama de Estado}

Diagramas de estado podem ser utilizados para visualizar e avaliar o fluxo lógico de um sistema. Eles exibem a forma pela qual informações, ações e eventos são transmitidos entre todas as entidades relacionadas a uma funcionalidade específica.

A seguir, são apresentados os diagramas de sequência de algumas das funcionalidades chave do projeto Greenhouse.
