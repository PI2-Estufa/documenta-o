\section{Confecção Ponte H}
A ponte H foi feita primeiramente em uma protoboard a fim de ser validada sua eficiência. Em seguida, o sistema foi transferido para a placa furada. O resistor utilizado apresentou mudança pois o item não estava disponível para a venda. Os resistores utilizados foram os 4 resistências de 560$\Omega$. Dessa forma:  

\begin{center}
	
	$\frac{3,3 - 0,7}{560}$ = $l_{b}$\\
	$l_{b}$ = 0,0046\textit{A}\\
	$I_{ce}$ = 1000 $\Omega$ * 0,0046 = 4,6 \textit{A}
	
\end{center}

\begin{itemize}

	\item $l_{b}$ = corrente que aciona o transistor;
	\item $I_{ce}$ = corrente disponível à carga.
	
\end{itemize}

O resultado final é mostrado a seguir:

\begin{figure}[H]
	\centering
	\includegraphics[width=13cm]{figuras/ponteHpronta.png}
	\caption{Ponte H confeccionada.}
	\label{ponte H pronta}
\end{figure}

Os integrantes que confeccionaram a placa tiveram dificuldades com a manipulação dos componentes, por não possuírem prática alguma com circuitos. Para posteriores trabalhos, é recomendado o uso de fios com bitolas maiores a fim de manter a segurança da ponte H.

\section{Confecção do plantário}

No sistema que está foi projetado para estufa, a área disponível para a locação da plantário era restrita a 50cmx50cm. Visto que esse espaço ainda seria compartilhado com os rolamentos da gaveta, a estrutura foi confeccionada em 3 tubos de PVC de 75mm de diâmetro e 40cm de comprimento. Além disso, foram feitos 2 furos em cada cano, com auxílio de serra-copo de 50mm de diâmetro, destinados a alocar as alfaces (as hortaliças do projeto). Os canos foram pintados de preto a fim de evitar o desenvolvimento de fungos e bactérias. O resultado é mostrado a seguir:

\begin{figure}[H]
	\centering
	\includegraphics[width=13cm]{figuras/pvc.png}
	\caption{Plantário confeccionado com canos PVC. Fonte própria}
	\label{pvc}
\end{figure}

O sistema deve contar com uma bomba que eleve a água do reservatório para 40 cm acima do mesmo, onde estarão os canos. Os canos estarão dispostos com uma angulação que permite que o fluído percorra o cano pela ação da gravidade. Assim, não é preciso fazer cálculos de perdas de cargas das raízes do substrato. A potência pelo fluxo mássico é dado pela variação de energia na entrada e saída da bomba. Como representado a seguir, há a variação de energia de pressão, energia cinética e energia potencial:

\begin{center}
	${\displaystyle\frac{W}{m} = (\frac{P2}{\rho} + \frac{V2^2}{2} + gH2) - (\frac{P1}{\rho} + \frac{V1^2}{2} + gH1)}$	
\end{center}

Onde, 

$W$ = potência consumida (W/s);

$m$ = fluxo mássico (kg/s);

$P2$ = pressão no ponto 2;

$\rho$ = massa específica da água (kg/m³);

$V2$ = velocidade no ponto 2 (m/s)

$H2$ = altura no ponto 2 (m);

$g$ = gravidade (m²/s);

$P1$ = pressão no ponto 1;

$V1$ = velocidade no ponto 1 (m/s);

$H1$ = altura no ponto 1 (m);

Considera-se que não há variação de altura considerável entre a entrada e saída da bomba. Assim, não há variação de energia potencial. Considera-se também que não há variação de energia cinética no sistema (V2 aproximadamente igual a V1). Nota-se, então, que o sistema fornecerá apenas energia de pressão, essa mesma que torna possível a elevação da coluna de água. Dessa forma:

\begin{center}
	\large
	${\displaystyle \frac{W}{m} = \frac{\Delta P}{\rho}}$
\end{center}

A variação de energia de pressão é dada por: 

\begin{center}
	\large
	${\displaystyle \Delta P = f \frac{L}{D} \frac{1}{2} \frac{Q^2}{A^2} \rho}$
\end{center}

Onde,

$f$ = fator de atrito;

$L$ = comprimento do escoamento (m) ;

$D$ = diâmetro;

$\rho$ = massa específica da água (kg/m³);

$A$ = área do escoamento ($m^2$);

$Q$ = vazão volumétrica ($m^3$/s).

Ao dividir a equação pela massa específica e gravidade, tem-se a altura manométrica (m):

\begin{center}
	\large
	${\displaystyle \Delta H = f \frac{L}{D} \frac{1}{2} \frac{Q^2}{A^2} \frac{1}{g}}$
\end{center}

É possível calculá-lo também a partir da equação de Coolebroke - White (1939). O fator de atrito de alguns materiais já são tabelados. A tabela a seguir dispõe de alguns dele:

\begin{figure}[H]
	\centering
	\includegraphics[width=13cm]{figuras/fatores_atrito.png}
	\caption{Fatores de atrito tabelados. Fonte: www.pipelife.com}
	\label{fatores_atrito}
\end{figure}

O material da tubulação de escoamento selecionada foi o polietileno, cujo fator de atrito é de 0,0015mm. O diâmetro do tubo é de 0,005m. A imagem a seguir o reporta já conectado aos tubos. 

\begin{figure}[H]
	\centering
	\includegraphics[width=8cm]{figuras/tubos_plantario.png}
	\caption{Tubos do plantário. Fonte própria}
	\label{fatores_atrito}
\end{figure}

Assim:

\begin{center}
	\large
	${\displaystyle \Delta H = 0,00015 \frac{0,4}{0,005} \frac{1}{2} \frac{Q^2*16}{\Pi^2 * (0,005)^4*10}}$
\end{center}

Isolando o termo da vazão e utilizando 0,4m como altura manométrica. Encontra-se Q= 5,103.$10^{-5}$ $m^{3}$/s ou 183,72 L/h. Assim, a bomba para o sistema de cada caneleta deve ter, no mínimo, esses parâmetros.
O dimensionamento da bomba não condiz com o que realmente será utilizado, pois a produção de bombas são tabeladas. Cada fornecedor de bombas trabalha com um rendimento específico do seu produto. O fornecedor de bomba Sarlobetter, por exemplo, traz a curva característica de seus produtos. Nota-se que, para trabalhar na vazão encontrada. A bomba ideal é a S300 (300L/h) visto que com o aumento da coluna a ser vencida, a vazão decai e a bomba deixa de operar com sua vazão de projeto.  

\begin{figure}[H]
	\centering
	\includegraphics[width=13cm]{figuras/coluna.png}
	\caption{Curva característica da moto bomba Sarlo  Fonte: <http://www.sarlobetter.com.br/aquarios/bombas/linha-sarlo/s-90/manual.pdf>}
	\label{coluna}
\end{figure}

A fim de reduzir os custos, serão utilizadas bombas que os integrantes já possuíam. A primeira bomba é de 160L/h e 3,8W e a segunda de 240L/h e 4W, que já foram devidamente testadas. A terceira será comprada após a instalação das outras duas no sistema. A imagem a seguir retratada as duas bombas:

\begin{figure}[H]
	\centering
	\includegraphics[width=8cm]{figuras/bombas.png}
	\caption{C Bombas de água. Fonte própria.}
	\label{coluna}
\end{figure}

O reservatório deverá conter também um compressor de ar com a finalidade de auxiliar no processo de dissolução dos nutrientes. Como seu fim é apenas mixer, sua seleção foi feita a partir do menor preço de mercado. O compressor que será utilizado é mostrado a seguir:

\begin{figure}[H]
	\centering
	\includegraphics[width=8cm]{figuras/compressor.png}
	\caption{Compressor de ar. Fonte própria.}
	\label{compressor}
\end{figure}


\subsection{Confecção da ventilação}

Os coolers já foram instaladas na estrutura, na parte superior e em sentidos opostos, a fim de garantir a circulação de fluído no interior e para fora.

\begin{figure}[H]
	\centering
	\includegraphics[width=8cm]{figuras/ventilador.png}
	\caption{Ventilador instalado na estrutura. Fonte própria.}
	\label{ventilador}
\end{figure}

O volume total do ambiente interno da estufa é dado por:

\begin{center}
	\large
	${\displaystyle V = b^2 * h = (0,5^2) * 0,7 = 0,175 m^3}$
\end{center}

Onde,

$V$ = volume total (m³);

$b$ = aresta da base quadrada (m);

$h$ = altura (m)

Dessa forma, o primeiro ventilador recicla esse mesmo volume em aproximadamente 16 segundos e o segundo o faz em aproximadamente 9 segundos.  
