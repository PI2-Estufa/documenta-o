\section{Ponte H}
Os motores de corrente contínua trabalham nos dois sentidos de rotação quando invertidas suas polaridades. A ponte H é um circuito que tem a função de controlar esse sentido de rotação do motor a partir da inversão de sua polaridade. O exemplo a seguir, ilustra uma ponte H composta de quatro transístores que trabalham em pares nas diagonais. Basicamente, quando se aciona uma chave tem-se 12V e a outra chave-par leva o terra (0V) para o motor. A utilização de transístores NPN e PNP é aconselhável para evitar uma perda de tensão maior entre eles, dessa forma, a carga (motor) fica sempre ligado aos coletores dos transístores.     

\begin{figure}[H]
	\centering
	\includegraphics[width=5cm]{figuras/ponteH.png}
	\caption{Representação ponte H}
	\label{ponte H}
\end{figure}

Para o projeto Green House, será necessária a confecção de uma ponte H para controle da rotação do motor. Foi realizada uma simulação no Proteus®, afim de garantir a tensão de 12V para carga, a funcionalidade do circuito e demonstrada também a entrada do controlador de operações (representado pelos 3,3V):

\begin{figure}[H]
	\centering
	\includegraphics[width=7cm]{figuras/simulacaoPonteH.png}
	\caption{Simulação da ponte H com acionamento em Q1 e Q4. Fonte: Proteus*.}
	\label{simulação ponte H}
\end{figure}

\begin{figure}[H]
	\centering
	\includegraphics[width=7cm]{figuras/simulacaoPonteH2.png}
	\caption{Simulação da ponte H com acionamento em Q2 e Q3. Fonte: Proteus*.}
	\label{simulação ponte H 2}
\end{figure}

O microcontrolador que será utilizado é uma Raspberry que trabalha com 3,3V e tem limitação na corrente de 50mA, quando utilizadas todas as suas entradas. Assim, para que essa restrição seja limitada, no nó de conexão entre o transístor e o controlador, a corrente foi limitada a 0,005mA. Para isso ocorrer, é necessário o uso de resistências de 520$\Omega$, como demonstrado a seguir:

\begin{center}
	
$\frac{V_{cc} - V_{be}}{R}$ = $l_{b}$ \\
$\frac{3,3 - 0,7}{R}$ = 0,005\\
R = 520$\Omega$\\
$I_{ce}$ = 1000 $\Omega$ * 0,005 = 5 \textit{A}

\end{center}

Onde,
\begin{itemize}

	\item Ib = corrente que aciona o transistor;
	
	\item R = resistências que serão utilizadas para controle da corrente Ib;
	
	\item $I_{ce}$ = corrente disponível à carga;
	
	\item ${V_{cc} - V_{be}}$ = diferença de tensão entre o controlador e o transístor.
	
	\item A corrente $I_{ce}$ é multiplicada por 1000 devido ao transistor TIP. Sendo ela fornecida pela fonte que será construída mais adiante.
	
\end{itemize}

Para a construção dessa ponte H, serão necessários:
\begin{itemize}

\item 2 transístores Darlington TIP 120;
\item 2 transístores Darlington TIP 125;
\item 4 resistências de 520$\Omega$;
\item 1 placa furada;
\item 3 Borne Conector 2 vias - entradas PCI

\end{itemize}

\section{Plantário}

Na hidroponia, o cultivo das hortaliças é feito em um lugar pequeno e que não gera resíduos no local (como quando utilizado areia ou terra). De acordo com Silva et al. [1], do grupo de fruticultura da Universidade Federal de Uberlândia, o pioneiro na aplicação da técnica de hidroponia foi Allen Cooper, no Glasshouse Crop Research Institute, na Inglaterra, em 1965. Cooper relatava que “a espessura do fluxo da solução nutritiva que passa através das raízes das plantas deve ser bastante pequeno (laminar), de tal maneira que as raízes não ficassem totalmente submergidas, faltando-lhes o necessário oxigênio” [1]. No Brasil, o método é amplamente difundido por meio de estruturas de PVC, que alocam as hortaliças e por onde flui água, que é oriunda de um reservatório e se destina ao mesmo reservatório após o caminho do plantio ser percorrido. A fim de ocupar o espaço disposto de forma otimizada, trabalhou-se para que o plantário tivesse a disposição semelhante com a reportada na imagem a seguir:

\begin{figure}[H]
	\centering
	\includegraphics[width=13cm]{figuras/plantario.png}
	\caption{Plantário. Adaptado de: <http://www.ecoeficientes.com.br/o-que-e-hidroponia/hidroponia-2/}
	\label{plantario}
\end{figure}

De acordo com Silva et al. [1], a vazão ideal no para uma estrutura hidropônica está entre 1,5 litro/minuto e 2,0 litros/minuto por canaleta de cultivo.

\section{Ventilação}
Estufas são espaços fechados a fim de não sofrerem as variações de temperatura, umidade e outros fenômenos naturais. Visto que a hortaliça da Green House é definida como a alface e sua faixa de temperatura para cultivo é consideravelmente extensa (10 a 24ºC) [2], não há a necessidade de um método para aumento de temperatura e, sim, resfriamento. Com intuito de unir duas soluções, ventilação e controle de temperatura, foi definido o uso de coolers para reciclar o ar no interior da estufa, auxiliando no ajuste de umidade e na variação de temperatura. A seleção do cooler envolve apenas a variável tempo, já que o fluxo de ar a ser reciclado variará de acordo com a vazão do aparelho. A fim de poupar gastos, foram selecionados dois coolers que os integrantes já detinham para entrada de fluxo e saída de fluxo (exaustor), ambos de 12V DC. Um deles é da marca Leadership e funciona com as seguintes especificações:

\begin{itemize}
	\item Corrente: 0,12A;
	\item Dimensões: 80x80x25mm;
	\item Vazão volumétrica: 40,32 m³/h;
	\item Velocidade: 2200 RPM  
\end{itemize}

O segundo ventilador é da marca Adasa e suas condições de operação são:

\begin{itemize}
	\item Corrente: 0,08A;
	\item Dimensões: 120x120x25mm;
	\item Vazão volumétrica: 75,9 m³/h;
	\item Velocidade: 1400 RPM 
\end{itemize}

